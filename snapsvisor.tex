\chapter{Snapsvisor}
\section{Livet är härligt}
\textit{Melodi: Röda kavalleriet}
\vspace{2mm}\\
Livet är härligt, (hej)\\
Tavaritj, vårt liv är härligt.\\
Vi alla våra små bekymmer glömmer\\
när vi har fått en tår på tanden - Skål!\\
\\
Tag dig en vodka, (hej)\\
Tavaritj, en liten vodka.\\
Glasen i botten vi tillsammans tömmer,\\
det kommer mer efter hande - Skål!

\section{Festen kan börja}
\textit{Melodi: Vårvindar friska}
\vspace{2mm}\\
Festen kan börja, ingen får sörja,\\
här finns det både brännvin och mat.\\
Helan skall tömmas, sorgerna glömmas,\\
ingen får vara tåkig kamrat.\\
Klappa mitt hjärta, fröjdas min själ,\\
nubbar serveras genast nåväl.\\
Nu tar vi supen öppna på strupen,\\
gästernas välkomstskål.

\section{Denna thaft}
\textit{Melodi: Helan går}
\vspace{2mm}\\
Denna thaft,\\
är den bästa thaft thythemet haft.\\
Denna thaft,\\
är den bästa thaft dom haft.\\
Och den thom inte har non kraft\\
han dricka thall av denna thaft.\\
Denna thaft\\
till landth, till sjöth, till havth.

\section{Hell and gore}
\textit{Melodi: Helan går}
\vspace{2mm}\\
Hell and gore,\\
Chung hop father Allan ley\\
Hell and gore,\\
Chung hop father Allan ley\\
For handsome in the hell and tar\\
and hell are in the half and four\\
Hell and gore,\\
Chung hop father Allan ley

\section{Mera brännvin i glasen}
\textit{Melodi: Internationalen}
\vspace{2mm}\\
Mera brännvin i glasen,\\
mera glas på vårt bord,\\
mera bord på kalasen,\\
mera kalas på vår jord.\\
\\
Mera jordar kring månen,\\
mera månar kring Mars,\\
mera marscher till Skåne,\\
mera Skåne, Gud bevars, bevars, bevars!

\section{Månvisa}
\textit{Melodi: Mors lilla Olle}
\vspace{2mm}\\
En gång i månan är månen full,\\
aldrig jag sett honom ramla omkull.\\
Full av beundran hur mycket han tål,\\
höjer jag glaset och dricker hans skål.

\section{Risjerk}
Nu går vi till Risjerk och tager oss en sup.\\
Det gör så gott i magen att få en liten sup.\\
$||:$ Ja, jag vågar liv och död\\
att på mig går ingen nöd\\
så länge det finns brännevin\\
och flickor i överflöd. $:||$

\section{Om cykla}
\textit{Melodi: Väva Vadmal \vspace{5mm} Text: Povel Ramel}
\vspace{2mm}\\
Man cyklar för lite,\\
man röker för mycket \\
och man är fasen så liberal\\
när det gäller maten och spriten.\\
Jag borde slutat för länge sedan,\\
men denna sup är så liten.\\
Vad tjänar att hyckla? \\
Tids nog får man cykla!

\section{Röda havet}
Vi gingo ned till Röda havet.\\
Vi lågo i där minst en kvart.\\
Men inte blev vi röda av'et,\\
men Röda havet det blev svart! \\
\\
$||:$ Men utav akvavit,\\
människan till kropp och själ\\
blir oskuldsfull och vit. $:||$

\section{Tårtsupen}
\textit{Melodi: Tårtan}
\vspace{2mm}\\
Skåne, Hallands, OP och Akvavit,\\
Gammeldansk och pomerans\\
hällt i en dunk med hemkörd sprit.

\section{Slutet sällskap}
Vi skåar för våra vänner\\
och dom som vi känner\\
och dom som vi inte känner,\\
dom skiter vi i!

\section{För att människan}
\textit{Melodi: Bär bä vita lamm}
\vspace{2mm}\\
För att människan ska trivas på vår jord\\
bör man ständigt ha på sitt smörgåsbord:\\
En stor, stor sup åt far,\\
en liten sup åt mor\\
och två små droppar åt lille, lille bror.

\section{Byssan lull}
\textit{Melodi: Byssan lull}
\vspace{2mm}\\
Byssan lull utav vinet blir man full\\
slipsen man doppar i smöret\\
och näsan den blir röd\\
och ögonen får glöd\\
men tusan så bra blir humöret.

\section{En liten gåsapåg med Skåne}
Jag är en liten gåsapåg med Skåne.\\
En Skåne som ni vet är alltid god.\\
Fast priset på systemet som ett rån é\\
så porlar alltid Skåne i mitt blod.\\
På bordet framför mig det står en pärla\\
nu ser jag livets sanningar så klart.\\
Med ett glas uti var hand,\\
går jag in i dimmigt land\\
$||:$ där man ser det lite grann så $:||$ (3 ggr)\\
där från ovan.

\section{Jäsa, jäsa brännvin}
\textit{Melodi: Skära skära havre}
\vspace{2mm}\\
Jäsa, jäsa brännvin\\
vem skall mäsken dricka?\\
Pekka säger inte nej,\\
vad säger Pekkas flicka?\\
Phy phan, phy phan\\
jag får sån djävla hicka!

\section{Be Be vitamin}
\textit{Melodi: Bär bä vita lamm}
\vspace{2mm}\\
Be-Be- vitamin, finns i brännevin\\
Mången kalori, simmar däruti.\\
Helgdagssup åt far och\\
söndagskrök åt mor samt tre små\\
huttar åt lille lille bror.

\section{Brännvin är så jäkla gott}
\textit{Melodi: Karl-Alfred Boy}
\vspace{2mm}\\
Brännvin är jäkla gott,\\
blir bättre ju mer man fått.\\
Och går man i golvet,\\
så där framåt tolv-ett,\\
då slår man sig jäkla hårt.

\section{Den siamesiska tvillingen}
\textit{Melodi: Petter Jönsson}
\vspace{2mm}\\
Det var en tvilling av siamesiska slaget,\\
som ej tog färre än tvenne supar i taget.\\
han helan tog för att därmed hedra sin moder\\
med halvan retade han sin helnyktre broder.
