\chapter{Fina Visor}
\section{Du gamla, du fria}
\textit{Text och musik: Richard Dybeck}
\vspace{2mm}\\
Du gamla, du fria, du fjällhöga Nord\\
Du tysta, du glädjerika sköna\\
Jag hälsar dig, vänaste land uppå jord,\\
$||:$ Din sol, din himmel, dina ängder gröna. $:||$\\
\\
Du tronar på minnen från fornstora dar,\\
då ärat ditt namn flög över jorden.\\
Jag vet att du är och du blir vad du var.\\
$||:$ Ja jag vill leva, jag vill dö i Norden. $:||$
\newpage

\section{Kungssången}
\textit{Text: C.V.A Strandberg \hspace{5mm} Musik: Otto Lindblad}
\vspace{2mm}\\
Ur svenska hjärtans djup en gång \\
en samfälld och en enkel sång,\\
som går till kungen fram!\\
Var honom trofast och hans ätt,\\
gör kronan på hans hjässa lätt,\\
och all din tro till honom sätt,\\
du folk av frejdad stam!\\
\\
O konung, folkets majestät\\
är även ditt: beskärma det\\
och värna det från fall!\\
Stå oss all världens härar mot,\\
vi blinka ej för deras hot:\\
vi lägga dem inför din fot,\\
en kunglig fotapall.\\
\\
Men stundar ock vårt fall en dag,\\
från dina skuldror purpurn tag,\\
lyft av dig kronans tvång\\
och drag de kära färger på,\\
det gamla gula och det blå,\\
och med ett svärd i handen gå\\
till kamp och undergång!\\
\newpage
\noindent Och grip vår sista fana du\\
och dristeliga för ännu\\
i döden dina män!\\
Ditt trogna folk med hjältemod\\
skall sömma av sitt bästa blod\\
en kunglig purpur varm och god,\\
och svepa dig i den.\\
\\
Du himlens Herre, med oss var,\\
som förr du med oss varit har,\\
och liva på vår strand\\
det gamla lynnets art igen\\
hos sveakungen och hans män.\\
Och låt din ande vila än\\
utöver Nordanland!
\newpage
\section{O gamla klang - och jubeltid}
\textit{Melodi: O alte Burschenherrlichkeit}
\vspace{2mm}\\
O, gamla klang och jubeltid,\\
ditt minne skall förbliva,\\
och än åt livets bistra strid\\
ett rosigt skimmer giva.\\
Snart tystnar allt vår yra skämt,\\
vår sång blir stum, vår glam förstämt;\\
O, jerum, jerum, jerum,\\
O, quae mutatio rerum!\\
\\
Var äro de som kunde allt,\\
blott ej sin ära svika,\\
som voro män av äkta halt\\
och världens herrar lika?\\
De drogo bort från vin och sång\\
till vardagslivets tråk och tvång;\\
O, jerum, jerum, jerum,\\
O, quae mutatio rerum!\\
\\
En tämjer börsens vilda fall\\
och köper våra papper,\\
en idkar maskinistens kall,\\
en mästrar volt så tapper,\\
en lagrar data i en fil,\\
en håller oss vid liv på RiL;\\
O, jerum, jerum, jerum,\\
O, quae mutatio rerum!\\
\\
Men hjärtat i en sann student\\
kan ingen tid förfrysa.\\
Den glädjeeld som där han tänt,\\
hans hela liv skall lysa.\\
Det gamla skalet brustit har,\\
men kärnan finnes frisk dock kvar,\\
och vad han än må mista,\\
den skall dock aldrig brista!\\
\\
Så slutet, bröder, fast vår krets\\
till glädjens värn och ära!\\
Trots allt vi tryggt och väl tillreds\\
vår vänskap trohet svära.\\
Lyft bägarn högt och svinga, vän!\\
De gamla gudar leva än\\
bland skålar och pokaler,\\
bland skålar och pokaler!
