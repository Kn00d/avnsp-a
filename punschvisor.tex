\chapter{Punschvisor}
\section{Punschen kommer}
\textit{Melodi: Vals ur Glada Änkan}
\vspace{2mm}\\
Punschen kommer,\\
punschen kommer,\\
ljuv och sval.\\
Glasen imma,\\
röster stimma\\
i vår sal.\\
Skål för glada minnen!\\
Skål för varje vår!\\
Inga sorger finnas mer\\
när punsch vi får.

\newpage
\section{Gul lyser solen}
\textit{Melodi: Lille Marlene }
\vspace{2mm}\\
Gul lyser solen,\\
Gul är flaggans kors.\\
Gul vajar säden,\\
Och gul är ängens pors.\\
Gul är kinesen, gul japan,\\
Och pyttens ägg är gul som fan.\\
Vår punsch är gul - så ta'n!\\
Vår punsch är gul - så ta'n!

\section{Punsch, punsch}
\textit{Melodi: Ritsch, ratsch, filibombombom}
\vspace{2mm}\\
Punsch, punsch, filibombombom\\
Filibombombom filibombombom\\
Punsch, punsch, filibombombom\\
Filibombombom filibombombom\\
Vi har ju både Cederlunds\\
och Carlshamns Flagg,\\
Grönstedts Blå, och lilla Caloric.\\
$||:$ Det duger ej med sodavatten,\\
sodavatten, sodavatten.\\
Det duger ej med sodavatten\\
Nej, ge oss mera punsch! $:||$

\newpage
\section{Min punsch}
\textit{Melodi: Min häst}
\vspace{2mm}\\
$||:$ Min punsch den är ljuv och ljusgul,\\
jag dricker den varm som kall.\\
Till ärtor med fläsk\\
och efter en bäsk,\\
när punschen smakar som bäst i magen $:||$

\section{Punschen gul}
\textit{Melodi: Vem kan segla}
\vspace{2mm}\\
Punschen gul uppå bordet står.\\
Punschen snällt på dig väntar.\\
Om din längtan till punschen går,\\
ta den utan att flämta.

\section{Punschen kommer hej, hej}
\textit{Melodi: Sommartider hej, hej}
\vspace{2mm}\\
$||:$ Punschen kommer hej, hej.\\
Punschen kommer.\\
Glasen dom imma,\\
röster dom stimma\\
å-å-å, inga sorger finnes mer. $:||$

\newpage
\section{Studiemedelsrondo}
\textit{Melodi: Ösa sand}
\vspace{2mm}\\
Vi dricker punsch\\
till lunch\\
när vi har fått avin.\\
Vi lunchar hela dagen\\
tills kassan gått i sin.\\
\\
Vi dricker punsch\\
till lunch.\\
Det är en medicin\\
som hjälper mot det mesta\\
och ger oss glada grin.\\
\\
Vi dricker punsch\\
ger stuns\\
åt tentamensamnestin.\\
Vi dricker för att trösta\\
och fira som små svin.

\newpage
\section{Vädjan till punschen}
\textit{Melodi: Sov du lilla videung}
\vspace{2mm}\\
Kom nu lilla punschen min.\\
följ nu efter supen.\\
Snart skall du åka in\\
ner igenom strupen,\\
till mitt stora magpalats,\\
där det finns så mycket plats.\\
Kom nu lilla punschen.\\
Följ nu efter supen.

\section{Sjung punschens lov}
\textit{Melodi: Plaisir d'Amour \hspace{5mm} Text: Laurent \& Walther (1982)}
\vspace{2mm}\\
Sjung punschens lov\\
Här kommer en gyllene kopp\\
Så drick nu ut det som finns\\
ja drick botten opp\\
\\
Du lömska dryck\\
Din vidrigt sliskiga smak\\
Med stank så hemsk att man nästan\\
blir dynghög-rak\\
\\
Nej, livet är punsch\\
En härlig styrka och doft\\
Så i min grsv lägg en flaska punsch\\
vid mitt stoft\\
\\
Usch fy så hemskt\\
Ska graven få samma odör\\
Som Carlshamns... hellre jag tror\\
jag i Skåne dör\\
\\
Fem kryddors arom\\
Har gjort dig ganska unik\\
Drick punsch min vän och sen blir\\
du dig aldrig lik\\
\\
Sjung punschens lov\\
Här kommer en gyllene kopp\\
Så drick nu ut det som finns\\
ja drick botten opp\\
\\
\textit{Lämpligen kan man sjunga vers 1 \& 5.}

\section{Ärtor utan punschen}
\textit{Melodi: Kaffe utan grädde}
\vspace{2mm}\\
Ärtor utan punschen\\
är som kaffe utan konjak,\\
och kaffe utan konjak\\
smakar inte särskilt bra.\\
Vi kann försaka whiskyn\\
och leva utan vodka,\\
men inte utan punschen\\
eller konjak vill vi va'.

\section{Sista punschvisan}
\textit{Melodi: Auld lang syne}
\vspace{2mm}\\
När punschen småningom är slut\\
Och vår flaska blivit tom,\\
Så vänder vi den upp och ner\\
Tills dess inget rinner ut.\\
\\
$||:$ Så slickar vi, så slickar vi,\\
Bad utanpå och i.\\
Och finns där ändå något kvar\\
får det va' till sämre dar $:||$
