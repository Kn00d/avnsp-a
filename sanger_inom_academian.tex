\chapter{Sånger inom Academian}
\section{Arbetslunchvisan}
\textit{Melodi: Kan du vissla Johanna}
\vspace{2mm}\\
Svinga bägaren vänner för nu är det fest\\
Academian skål här passar vi bäst\\
Här är gänget med fötter som tar sig en hutt\\
och håller sen på tills brännevinet är slutt\\
\\
Våra hängslen är breda vår mage av is\\
tar festerna slut kommer vanligen en kris\\
Med det bästa från källarn vi fyller våra glas\\
Är alltid de sista som ger upp ett kalas\\

\section{Första Snapsvisan}
\textit{Melodi: Studentsången}
\vspace{2mm}\\
Höjom pokalen vänner i fest\\
Academian har ånyo församlats\\
Fötterna pryda vår mörkblå väst\\
Väl vi hälsar var ärad gäst\\
\\
Le var sann gourmé, le du som törstig är\\
Le för att du vet, att det nalkas en kväll\\
utav sånger och brännvin och helfestlighet\\
Syndaflod har vi inget emot,\\
Töm nu glaset i vänsterfot\\
\section{Den i:te snapsvisan}
\textit{Text: Henrik Berg \hspace{5mm} Melodi: Stars and stripes}
\vspace{2mm}\\
Vänd på slipsen nu vän,\\
låt oss ta en tår igen.\\
Sjung ur strupen rakt ut,\\
innan festen tagit slut.\\
\\
Stämman ljuv och unison,\\
när Arbetslunchen tar ton.\\
Med stil och med smak,\\
under Ryds Herrgårds tak.\\
\\
Här hos Academian,\\
För dryga hundratjugonian,\\
ska vi festa natten lång.\\
Töm ditt glas, sätt igång!\\
\section{Än en gång däran...}
Än en gång däran bröder!\\
Än en gång däran!\\
Följom den urgamla seden.\\
itill sista man bröder,\\
intill sista man,\\
trotsa vi hatet och vreden.\\
Blankare vapen sågs aldrig i en här,\\
än dessa glasen kamrater i gevär!\\
Än en gång däran,\\
Svenska hjärtans djup här är din sup!\\
*\\
Livet är så kort bröder!\\
Livet är så kort!\\
Lek det ej bort, nej var redo.\\
Kämpa mot allt torrt bröder,\\
Kämpa mot allt torrt.\\
Tänk på de tappra som skredo\\
fram utan tvekan i floder av Champagne,\\
styrkta från början av brännvin från vårt land!\\
Kämpa mot allt torrt,\\
Svenska hjärtans djup här är din sup!\\
\vspace{\fill}
\\
\textit{*En snaps mellan verserna, enligt Evert Taube!}
