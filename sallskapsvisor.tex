\chapter{Sällskapsvisor}
\section{Tårtan}
Socker, grädde, nötter och mandelflarn,\\
och så sist men inte minst\\
en liten ros av marsipan.\\
\\
Smörkräm, krikon, snabbkräm och gott gelé.\\
Frasses deg och en flaska saft\\
och så en liten klick med sylt.

\section{Tåget}
Plingelingeling, nu ska tåget gå\\
ut i vida världen.\\
Den som femtio öre har\\
får följa med på färden.\\
Se... See...\\
nu löser hon biljetten\\
nu stiger hon pååå\\
och sen kan tåget gå!

\section{Brev från kolonien}
\textit{Text & musik: Cornelis Wreeswijk}
\vspace{2mm}\\
Hejsan morsan, hejsan stabben!\\
Här e brev från älsklingsgrabben.\\
Vi har kul på kolonien,\\
vi bor tjugoåtta gangstergrabbar i en...\\
\\
...stor barack med massa sängar.\\
Kan ni skicka mera pengar\\
För det vore en god gärning,\\
jag har spelat bort varenda dugg på tärning.\\
\\
Här e roligt vill jag lova\\
fastän lite svårt att sova.\\
Killen som har sängen över mej\\
han vaknar inte han när han behöver, nej!\\
\\
Jag har tappat två framtänder\\
för jag skulle gå på händer\\
när vi lattjade charader\\
så när morsan nu får se mej får hon spader.\\
\\
Uti skogen finns baciller\\
men min kompis han har piller\\
som han köpt utav en ful typ\\
och om man äter dom blir man en jättekul typ.\\
\\
Föreståndarn han har farit,\\
han blir aldrig vad han varit,\\
för polisen kom och tog hand\\
om honom förra veckan när vi lekte skogsbrand.\\
\\
Uti skogen finns det rådjur,\\
i baracken finns det smådjur\\
och min bäste kompis Tage\\
han har en liten fickkniv inuti sin mage.\\
\\
Honom ska dom operera.\\
Ja, nu vet jag inget mera.\\
Kram och kyss och hjärtligt tack sen\\
men nu ska vi ut och bränna grannbaracken!

\section{Var bor du, lilla råtta?}
\textit{Text & musik: B G Hallquist}
\vspace{2mm}\\
Var bor du lilla råtta?\\
I din hatt.\\
Vad gör du klockan åtta?\\
Jagar katt.\\
Hur många ungar har du?\\
Sjuttiotvå.\\
Hur mår din gamla far, då?\\
Si och så.\\
\\
Vad gör din fru Louisa?\\
Steker glass.\\
Vad vill du ha att spisa?\\
Korv och jazz.\\
Vad vill du ha att dricka?\\
Smultronbål.\\
Vad har du i din ficka?\\
Fyra hål.

\section{Min häst}
\vspace{2mm}\\
Min häst den har gula öron\\
och vita små skor av trä.\\
Röda strumpor till knät,\\
han är ganska fet,\\
och under magen har han blå små vingar.

\section{Min lilla häst}
\textit{Text & musik: Gullan Bornemark}
\vspace{2mm}\\
På fyra ben går den som jag gillar allra mest,\\
gillar, gillar, gillar allra bäst.\\
En skäppa full med havre får min lilla häst,\\
är min lilla häst.\\
\\
Vad du är söt, min kära lilla ponny.\\
Vad du är snäll, min kära lilla häst.\\
Du säger ingenting min kära lilla ponny,\\
men du är den jag gillar bäst, min häst.

\section{Professor Baltazar}
\vspace{2mm}\\
$||:$ Balt, Balt, Baltazar.\\
Balt, Balt, Baltazar.\\
Balt, Balt, Baltazar, Baltazar. $:||$

\section{Telefonen}
\textit{Text \& musik: Gullan Bornemark}
\vspace{2mm}\\
Jag har en telefon, som går upp i det blå\\
och när nån ringer på, så svarar jag som så\\
Hallå, hallå, hallå, vad är det som står på?\\
Jo, det är Frälsningsarmén som ringer på.\\
\\
Ja, jag kastar mina sorger bakom min rygg,\\
jag ser dom inte mer, jag ser dom inte mer.\\
Jag kastar mina sorger bakom min rygg,\\
jag ser dom inte mer!\\
\\
Jag är så lycklig, jag är så lycklig,\\
måndag, tisdag, onsdag, torsdag,\\
fredag, lördag, sönadag.\\
Jag är så lycklig, jag är så lycklig,\\
hela veckan lång!\\
\\
Nu tar vi spårvagnen, upp till himmelen,\\
med Gud som kör och Jesus som konduktör.\\
Det går så lätt, lätt, lätt\\
när man har fribiljett\\
för jag har blivit frälst, halleluja!

\section{En kulen natt}
\textit{Text \& musik: Gullan Bornemark}
\vspace{2mm}\\
En kulen natt-natt-natt\\
min båt jag styrde\\
på havets vågade-vågade-våg\\
så skummet yrde.\\
Och vart jag sågade-sågade-såg\\
på havets vågade-vågade-våg\\
långt ner i djupete-pete-pete-pet\\
en fisk jag såg\\
och det var du!\\
\\
\textit{Rörelser:}\\
Styrde $\rightarrow$ Fatta den visuella ratten och styr\\
Våg $\rightarrow$ Cosinus rörelse\\
Yrde $\rightarrow$"Splätt"\\
Såg $\rightarrow$ Spana av\\
Djup $\rightarrow$ Peka nedåt\\
Du $\rightarrow$ Peka på någon

\section{Familjen krokodil}
\textit{Text \& musik: Margit Gergårdh}
\vspace{2mm}\\
I Niggerland, händer ibland\\
att man får se på Nilens svarta strand\\
Herr krokodil\\
Fru krokodil\\
och lilla krokodilen Lilleman\\
\\
Solen den skiner ner på negerlandet ner\\
Där ligga de och gäspa käkarna ur led\\
Herr krokodil\\
Fru krokodil\\
och lilla krokodilen Lilleman\\
\\
Herr krokodil, Fru krokodil\\
och lilla krokodilen Lilleman\\
De dra åstad,\\
vandra i rad\\
till gamla negerdoktorn Pillerman\\
\\
Krokodilen Lilleman, han skrek i himlens höjd\\
Men doktor Pillerman han grinade förnöjd;\\
Jag tror minsann\\
Gapa du kan!\\
Jo-jo, sa gamle doktor Pillerman\\
\\
Doktorn tar fram, högst alvarsam\\
en jätteflaska full med levertran\\
håll dig nu still\\
Herr krokodil\\
sa han, och käkens gångjärn smorde han\\
\\
Herr krokodil slog gapet samman med en smäll\\
och det blev slut på krokodilebarnets gnäll\\
Gäspning man kan\\
bota minsann,\\
med levertran sa doktor Pillerman

\section{I ett hus}
I ett hus vid skogens slut,\\
liten tomte tittar ut.\\
Haren skuttar fram så fort,\\
klappar på dess port.\\
Hjälp ack hjälp ack hjälp du mig\\
annars skjuter jägarn mig.\\
Kom och kom i stugan in,
räck mig handen din\\

\section{Under en filt i Madrid}
\textit{Ur revyn Cyklar med Galenskaparna och After Shave}
\vspace{2mm}\\
Under en filt i Madrid\\
Ligger en flicka på glid\\
Tittar på mannen bredvid\\
Under en filt i Madrid\\
\\
Bakom ett berg i Genève\\
Där får en moder ett brev\\
Från hennes dotter på glid\\
Under en filt i Madrid\\
\\
Framför en stolpe i Bonn\\
Sitter det nu inte nån\\
Endast en tom La Garonne\\
Framför en stolpe i Bonn\\
\\
Men där i vindarnas drev\\
Fladdrar ett brev från Genève\\
Postat nån gång i Bretagne\\
Doftande billig champagne\\
\\
På en bordell i Borås\\
Smörjer en herre sitt krås\\
Bakom ett skjul i Tasjkent\\
Där står ett fönster på glänt\\
\\
Någon har kastat ett skal\\
Genom en jak i Nepal\\
Ingenting är som det skall\\
Solen är blott en marschall\\
\\
Själv är jag blott en kostym\\
Mamma är bara parfym\\
Pappa förspiller sin tid\\
Under en filt i Madrid\\
\\
Under ett lakan i Prag\\
Ligger en kvinna och jag\\
Sängen är full av resår\\
Sången jag sjunger är svår\\
\\
Omöjlig att komma ur\\
Jag vet då fan inte hur\\
Orden får snart inte rum\\
Jag får väl sjunga mig stum\\
\\
Tonerna trängs i min gom\\
Sätt mig på tåget till Rom\\
Låt mig få sluta min tid\\
Under en filt i Madrid

\section{Apans sång}
\textit{Ur Djungelboken}
\vspace{2mm}\\
Jag kungen é över alla här\\
under trädens gröna höjd.\\
Jag har nått opp till högsta topp,\\
men ännu är jag ej nöjd.\\
Jag vill ju va' en man en männska\\
och kunna allt du kan.\\
Jag vill ej längre apa mig.\\
Jag vill bara va' en man.\\
\\
Åh, oubidou, jag vill ju va' som du-u.\\
Jag vill se ut som du, gå som du, du-u.\\
Det vill jag nu-u, ett djur som ja-a-ag\\
det lär sig nog att bli en människa.\\
\\
- Nu till din del av vårt avtal.\\
Du säg mig hur gör man upp eld!\\
- Jamen, jag vet ju inte hur man gör upp eld.\\
\\
Försök inte lura mig gosse.\\
Jag inga konster tål.\\
Att känna till hur eld blir till\\
är mina drömmars mål.\\
Din hemlighet vill jag veta.\\
Höörnu, säg hur går det till?\\
För då blir jag visst\\
en man till sist.\\
Det är just vad jag vill.\\
\\
Åh, obidou...

\section{Kalles klätterträd}
I stan bor en kille som heter Kalle.\\
Kalle har ett träd som han kallar sitt eget.\\
Han ligger där uppe och drömmer och fantiserar\\
om allt möjligt och om Emma.\\
\\
Emma finns bara i fantasin \\
men hon är fin tycker Kalle.\\
\\
Under trädet sitter morfar och läser tidningen.\\
Morfar tycker om att läsa om allt nytt som har hänt.\\
Men Kalle han gillar att mest ligga stilla\\
i toppen av trädet och vaja med vinden.

\section{Jag vill ha blommig falukorv}
\textit{Text \& musik: Hasse Alfredsson}
\vspace{2mm}\\
Jag vill ha blommig falukorv till lunch, mamma.\\
Nåt annat vill jag inte ha.\\
Jag hatar tomaten och fisken och spenaten\\
och plättarna med lingonsylt\\
\\
Fläsk - har vi för ofta\\
Lamm - smakar som kofta.\\
Biff med lök är riktigt läbbigt.\\
Jag vill ha blommig falukorv till lunch, mamma.\\
Nåt annat vill jag inte ha.\\
\\
Jag vill ha blommig falukorv till lunch, mamma.\\
Nåt annat vill jag inte ha.\\
Nä, aldrig jag äter mer rotmos och potäter\\
och isterband och kalvkotlett.\\
\\
Pytt - det är för pyttigt.\\
Mjölk - det är för nyttigt.\\
Knäckebröd för hårt att tugga.\\
Jag vill ha blommig falukorv till lunch, mamma.\\
Nåt annat vill jag inte ha.

\section{Vi cyklar runt i världen}
\textit{Text \& musik: Ulf Dageby (Nationalteatern)}
\vspace{2mm}\\
Vi cyklar runt i världen\\
Vi spelar på gator och torg\\
Vi spelar på allt som låter\\
Ja, till och med på vår hoj\\
\\
Vi spelar för små hästar\\
Som bjuder oss på Toy\\
\\
Vi spelar för gubbar, spelar för gummor\\
Spelar för alla som vill ha skoj\\
Vi spelar för gubbar, spelar för gummor\\
Spelar för alla som vill ha skoj\\
\\
Ibland när vi står där och spelar\\
Så kommer det fram till oss\\
Sura gamla gnetar\\
Som inte gillar oss\\
\\
Då vill dom köra bort oss\\
Men vi kan inte slåss\\
Då spelar vi bort\\
Spelar vi väck dom\\
Spelar dom ända till Veskaförs!

\section{Sjörövar-Fabbe}
\textit{Text \& musik: Hasse Alfredsson}
\vspace{2mm}\\
Sjörövar Fabbe farfars far\\
är minsann en sju särdeles karl\\
kring alla hav han far och far,\\
tjohej hade littan lej.\\
Sjörövar yrket passar'n bra.\\
Det är bara att röva och ta,\\
och det sa, Fabbe, gillar ja\\
tjohej hade littan lej.\\
\\
Men då... vad står på?\\
Fabbe blev plötsligt blek och grå.\\
Oj då! Vad står på?\\
Oj oj oj oj oj oj oj...!\\
\\
Sjörövar Fabbe farfars far\\
är minsann en sjusärdeles karl\\
men han är sjösjuk i alla dar,\\
tjohej hade littan lej.\\
\\
Stormen ryter och åskan går,\\
havet brusar och seglen slår,\\
ner i kajutan fabbe går,\\
tjohej hade littan lej!\\
Kräks och svär och mår inte bra.\\
Bättre väder det vill ja ha,\\
annars, sa Fabbe slutar jag\\
tjohej hade littan lej.\\
\\
Men då... Vad står på?\\
Fabbe blir plötsligt blek och grå.\\
Oj då! Vad står på?\\
Oj oj oj oj oj oj oj...!\\
De'ä ingen mänska förstår\\
varför alltid så illa jag mår\\
bara båten guppar och går,\\
tjohej hade littan lej!\\
Sjörövar Fabbe farfars far\\
är minsann en sjusärdeles karl\\
men han är sjösjuk alla dar,\\
tjohej hade littan lej.

\section{Fantomens brallor}
Ingen har sett Fantomen utan kläder\\
klädd i pyjamas och stövlar av läder.\\
Han drar nog bort en rand där fram\\
när han kissar bakom trädets stam.\\
\\
\textit{Refr:}\\
\\
O, vandrande vålnad, kliar inte sviden\\
när du knegar i djungeln hela tiden?\\
Gör som Guran, skaffa dig en kjol,\\
det är bättre under Afrikas sol.\\
\\
Ingen har sett honom kavla upp ärmen\\
ljusblå lekdräkt i fukten och värmen.\\
Men han blev nog frusen om sin häck\\
om han satt i grottan alldeles näck.\\
\\
\textit{Refr.}\\
\\
Fantomen lättar inte på kalsongen\\
nej, han håller värmen stången.\\
Ibland tar han på sig ännu mer\\
när Mr Walker sig till staden beger.\\
\\
\textit{Refr.}\\
\\
Men ibland i grottan längst därinne\\
tar vålnaden fram sin vandrande pinne\\
Diana stack nog snart sin kos\\
om hon ej får smaka djungeljuice.\\
\\
\textit{Refr.}

\section{Barn av vår tid}
\textit{Text \& musik: Nationalteatern}
\vspace{2mm}\\
Vi är barn av vår tid\\
vi är barn av vår tid\\
är du rädd för ditt eget barn?\\
lilla mamma!\\
Våra tidsfördriv\\
våra tidsfördriv\\
är att slå pensionärer på käften!\\
Eller hur?\\
Lilla mamma.\\
\\
Vaktbolagen kommer snart\\
då blir det en jävla fart\\
dem har betalt för att jaga ungar!\\
gården är stängd för länge sen\\
snuten jagar tonårsgäng\\
natten är så hård\\
för betongens kungar!\\

Vi är barn av vår tid\\
vi är barn av vår tid\\
är du rädd för ditt eget barn?\\
lilla mamma!\\
\\
Thinnertrasan vandrar mellan husen\\
thinnertrasan tänder alla ljusen\\
thinnertrasan tar mig till ett annat land\\
där jag kan vara en höghus baby\\
säga till tjejen att, maybe...\\
så kan vi segla på molnen tillsammans\\
come on my darling\\
vi glömmer allt annat!\\
\\
farsan sitter hemma framför TVn\\
morsan sitter antagligen bredvid\\
lika bra att ta sig ner till EPA's torg\\
där kan vi låta betongen gunga!\\
gasa och sniffa och flumma!\\
där kan vi segla på molnen tillsammans!\\
come on my darling\\
vi glömmer allt annat!\\
\\
kom igen lilla Svensson\\
sätta hårt mot hårt!\\
det är våran stil\\
hatar du oss\\
så hatar vi dig\\
betong feeling!\\
\\
Ödmjuka bracka\\
som bugar för överheten\\
du skickar snuten på ungar\\
sen sover du snällt\\
i borgarsmeten.\\
Blev du rädd nu din fega fan?\\
För ditt eget barn!\\
\\
Jag vet att du sliter varje dag på fabriken\\
men om det ska gå ut över mig så blir jag besviken!\\
för vi är inte snälla\\
vi kraschar systemets fönster\\
inte husses hund\\
inga krypande svassande mönster.\\
Blev du rädd nu din fega fan?\\
för ditt egen barn!\\
\\
Det enda du kommer på\\
sy in din son på en kåk\\
vissla på en snutlakej\\
så kommer dem och hämtar mej\\
medan borgarasen flinar\\
åt hur jobbardräggen svinar!\\
Farsan tänk till!\\
Vad är det du vill?!\\
\\
DEM BURAR IN DINA EGNA BARN! \textit{x4}
