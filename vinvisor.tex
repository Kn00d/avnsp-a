\chapter{Vinvisor}
\section{Bordeaux, Bordeaux}
\textit{Melodi: I sommarens soliga dagar}
\vspace{2mm}\\
Jag minns än idag hur min fader\\
kom hem ifrån staden så glader\\
och rada’ upp flaskor i rader\\
och sade nöjd som så:\\
Bordeaux, Bordeaux!\\
\\
Han drack ett glas, kom i extas,\\
och sedan blev det stort kalas.\\
Och vi små glin, ja vi drack vin,\\
som första klassens fyllesvin,\\
och vi dansade runt där på golvet\\
och skrek så vi blev blå:\\
Bordeaux, Bordeaux!

\section{Feta Fransyskor}
\textit{Melodi: Marche Militaire av Franz Schubert}
\vspace{2mm}\\
Feta fransyskor som svettas om fötterna,\\
de trampar druvor som sedan ska jäsas till vin.\\
Transpirationen viktig é\\
ty den ge’ fin bouquet.\\
Vårtor och svampar följer me’\\
men vad gör väl de’?\\
För…\\
Vi vill ha vin, vill ha vin, vill ha mera vin,\\
även om följderna blir att vi må lida pin.\\
\textit{Flickor:} Flaskan och glaset gått i sin.\\
\textit{Pojkar:} Hit med vin, mera vin!\\
\textit{Flickor:} Tror ni att vi är fyllesvin?\\
JA! (fast större)

\section{I Frankrike dricks det viner}
\textit{Melodi: Eurovision-signaturen}
\vspace{2mm}\\
I Frankrike dricks det viner\\
när tyskarna dricker öl\\
underbart de mår.\\
Men svensken som dricker, svin är.\\
Oss svin emellan:\\
Tag en tår!

\newpage
\section{Inre dialog}
\textit{Melodi: An der Schönen Blauen Donau }
\vspace{2mm}\\
Jag vill inte ha\\
MERA VIN, MERA VIN!\\
Jag mår inte bra\\
MERA VIN, MERA VIN!\\
Om ni mig ger mer\\
MERA VIN, MERA VIN!\\
Ser jag er som fler\\
MERA VIN, MERA VIN!\\
min mage är sjuk\\
MERA VIN, MERA VIN!\\
Jag kan inte tänka så bra\\
så jag får väl vinet ta\\
HURRA!

\section{Sudda sudda}
\textit{Musik: Gullan Bornemark}
\vspace{2mm}\\
Sudda, sudda, sudda, sudda bort din sura min\\
med fyra jättestora bamseklunkar ädelt vin.\\
Munnen den ska sjunga och va gla'\\
för att den ska bli som den va'\\
vad häller du då bak det dolda flinet?\\
Vinet!\\
Som suddar, suddar bort din sura min.

\section{Undulaten}
\textit{Melodi: Med en enkel tulpan}
\vspace{2mm}\\
Jag är en liten undulat\\
som får så dålig mat\\
för dom jag bor hos\\
ja, dom jag bor hos\\
dom är så snåla.\\
Jag får ju fisk varenda dag\\
och det vill jag inte ha\\
jag vill ha rödvin,\\
jag vill ha rödvin och gorgonzola!\\
\\
Jag är en gammal dromedar\\
som inte har mycket kvar\\
och dom jag bor hos\\
ja, dom jag bor hos\\
dom är så sura!\\
Dom ger mig vatten i en hink\\
det tycker jag smakar pink\\
jag vill ha brännvin\\
jag vill ha brännvin\\
och Angostura!

\newpage
\section{Vårvinets lov}
\textit{Melodi: I sommarens soliga dagar}
\vspace{2mm}\\
Se, vinet det glimmar i glasen,\\
av vin blir det glatt på kalasen.\\
Sopranen, tenoren och basen\\
vid Bacchi hov\\
vill sjunga vinets lov.\\
\\
Därför ej dröj - pokalen höj\\
och dig med druvans saft förnöj.\\
En nektar som vi tycker om\\
och återverkar småningom.\\
I vårdagars roliga stunder\\
ett glatt och fylligt vin är melodin!

\section{Imsig vimsig}
\textit{Melodi: Imse vimse spindel}
\vspace{2mm}\\
Imsig vimsig blir man\\
utav lite vin.\\
Klättrar uppå stolen,\\
verkar piggelin.\\
Ramlar under bordet,\\
sussar en minut.\\
Vaknar av att vinet\\
i glaset har tagit slut.

\section{Franska vanor}
\textit{Text: Claes Laurent 1983}
\vspace{2mm}\\
Med franska vanor mår man ofta bra\\
och van i franska det är jag\\
man börjar lätt me lite champagne så här\\
du pain, du bör äta smör med cammebert\\
La France est un parti de l'Europe\\
Charles de Gaulle, Tricolore, tout le monde,\\
Citroën, Peugeot et Renault\\
Escargot, l'amour et la mer\\
Charmonix, Chourchevel et Val d'Isère\\
Mon Dieu, La Seine, ça va?\\
Cannes et Nice, Champs Elysées\\
Paris, Bordeaux, Versailles et Marseille\\
Et merde, trop de soleil\\
\\
Liberté, élegante, fraternité\\
On me dit; c'cest la verité\\
Pompidou, Michelin, oui je t'aime\\
Beignets abricots, coq au vin\\
Monseur, le fromage et alors\\
La tour Eiffel, boules soixante-dix-sept\\
L'arc de triomphe et la baguette\\
Borgogne, les Alpes ça m'est égale\\
Moulin Rouge à la Place Pigalle\\
Concorde et Debussy\\
Sacré-C\oe{}ur et Jaques Tati\\
Liberté, élegante, fraternité\\
Messieurdames a votre santé!

\section{Röd vitamin}
\textit{Melodi: My Bonnie}
\vspace{2mm}\\
Hur badar man bäst på en kurort?\\
Jo, om man har fyllt en bassäng\\
med vätskan som snart skall besjungas\\
när vi kommit fram till refräng;\\
\\
Rödvin, rödvin.\\
Rödvin är fin hälsokost, kost, kost.\\
Rödvin, rödvin.\\
Rödvin vår bästa flaskpost.\\
\\
Man får vitaminer från rödvin.\\
Man piggnar ju till på en gång,\\
när glaset har tömts uti botten\\
så stämmer vi upp till en sång.\\
\\
Rödvin, rödvin...
